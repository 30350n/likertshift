\section{Discussion}\label{sec:discussion}

\subsection{Research Questions}

\subsubsection{Q1: How do different methods to record cyclists' subjective experiences affect their
rating behavior?}

We failed to answer this question, as the evaluation of the quantitative data we recorded did not allow us to perceive any large differences.
Our best guess is that quantitative differences between methods are very subtle and require larger studies to be found.

\subsubsection{Q2: What are cyclists' preferences regarding methods to record their subjective experiences?}

Users highly prefer methods that require minimal additional work and appear invisible to other traffic participants.
While none of the methods we compared appear as particularly disturbing, especially to the users' ability to stay in control and cycle safely, \audiorecording was perceived as unpleasant, while the \mapping method exerted a higher mental demand on them.

\subsection{Future Work}

Future work could include the combination of multiple methods for recording \CSE, most notably hybrid methods, which rely on real-time data acquisition during cycling, but allow for modification after data collection is complete.
The easiest implementation of this would involve combining our \likertshift method with an adjustable map included in the smartphone app, similar to the \mapping method, but in a digital form.
This would allow for corrections, in situations where users performed rating too late or forgot them altogether.
Other improvements include the addition of active feedback to the device, for example, in the form of vibrations or sound.
This would allow for an even more intuitive rating process.
Finally, more sensors and connectivity options could be added to our \likertshift device, to make it usable standalone, without requiring a smartphone to actively record the data.
