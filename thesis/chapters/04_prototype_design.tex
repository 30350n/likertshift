\section{Prototype Design}\label{sec:prototype_design}

\subsection{Design Requirements}

To streamline the design of our prototype and to ensure its usefulness to other researchers in future work, we established the following design requirements:

\begin{enumerate}[label=\textsf{DRQ\arabic*}, left=1em .. 4.5em]
    \item \textbf{Intuitive}\label{drq:intuitive}\\
        Controlling the prototype device should be intuitive to the user.

    \item \textbf{Robust}\label{drq:robust} \\
        Using the prototype device should require minimal user intervention and maintenance. Users should not be able to easily break the device, and it should withstand external influences such as humidity and temperature variations.

    \item \textbf{Safe}\label{drq:safe}\\
        The prototype device should not compromise the user's safety.

    \item \textbf{Affordable}\label{drq:affordable}\\
        The total cost of all required components should not exceed \euro25.00.

    \item \textbf{Easy to Reproduce}\label{drq:easy_to_reproduce}\\
        The prototype device should be able to be manufactured and assembled using standard tools commonly available in most research laboratories.
\end{enumerate}

\noindent
In \autoref{subsec:discussion_design_requirements} we analyze how well our final prototype device meets these requirements based on the results gathered from our field study.

\subsection{Mechanical Design}

\subsubsection{Initial Design Decisions}

\subsubsection*{Continuous or Discrete Scale}

As discussed in \autoref{sec:motivation}, our prototype device should enable users to communicate their Travel Experience on a simple one-dimensional scale.
We need to decide whether this scale should be continuous or consist of discrete steps.

Theoretically, we would be able to capture more precise data with a continuous scale than with a discrete one.
However, since our use case is recording Travel Experience data - which is highly subjective - recorded values will vary significantly between different users.
So to obtain meaningful results, any recorded Travel Experience data must be averaged across a reasonably large pool of different users.
Therefore, a continuous input scale would not provide a substantial benefit for our specific use case.

This decision also directly influences the electromechanical design of the device.
Choosing to design a continuous scale would force us to either use a variable resistor (potentiometer) or a high resolution digital encoder.
In contrast, working with discrete steps would suggest the use of a rotary switch or multiple mechanical buttons.

Due to the missing benefit of a continuous scale and our previous decision to use a custom-built mechanism, we decided on a discrete scale, because we think this will be easier to implement using our chosen methods.

\cite{gesturing_on_the_handlebars} \cite{no_need_to_stop} \cite{text_me_if_you_can} \cite{brotate_and_tribike}

\begin{itemize}
    \item potentiometers vs. rotary switches vs. buttons
    \item brotate and tribike \cite{brotate_and_tribike}
    \item advantages/disadvantages of using our own hardware design (cost, reliability, availability, obsolete components)
\end{itemize}

\subsection{Electrical Design and Firmware}

\begin{itemize}
    \item BLE
    \item MCU choice (softdevices conformity)
    \item battery runtime analysis
\end{itemize}

\subsection{Frontend Design}

\begin{itemize}
    \item reusability
    \item BLE interfacing (communication "protocol")
    \item language/framework choice?
    \item UI design?
\end{itemize}
