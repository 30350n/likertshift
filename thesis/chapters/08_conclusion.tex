\section{Conclusion}\label{sec:conclusion}

In this work, we described the successful design and construction of the \likertshift, a prototype device that can be used to record cyclists' subjective experiences, as well as its evaluation against other state-of-the-art methods in a semi-naturalistic field-study.
Although quantitative analysis of recorded data revealed little to no statistical evidence supporting a measurable improvement in data quality using our developed \likertshift method, qualitative assessments of participants opinions regarding the different methods provided valuable insights.
They highlighted the perceived attractiveness, intuitiveness, and efficiency of using our physical device.
They also voiced concerns about the social acceptance of speech recording during cycling and the high effort and increased mental demand associated with retrospective mental mapping approaches that rely on the memorization of route segments, particularly for longer routes.
From our collected evidence, we see a great potential in using physical devices for recording data on cyclists' subjective experiences, especially in longer, more naturalistic field-studies.
