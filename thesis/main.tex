%%%%%%%%%%%%%%%%%%%%%%%%%%%%%%%%%_TEMPLATE_PACKAGES_%%%%%%%%%%%%%%%%%%%%%%%%%%%%%%%%%
\documentclass[
	a4paper,
	pagesize,
	pdftex,
	12pt,
	% twoside,    % + BCOR darunter: für doppelseitigen Druck aktivieren, sonst beide deaktivieren
	% BCOR=5mm,   % Dicke der Bindung berücksichtigen (Copyshop fragen, wie viel das ist)
	english,
	fleqn,
	final,
	]{scrartcl}
    
% PACKAGES FOR THE INSTITUTSVORLAGE
\usepackage[utf8]{inputenc}
\usepackage[english]{babel}
\usepackage[unicode=true,hidelinks]{hyperref} % label, references, websites and crosslinks in PDF
\usepackage{setspace} % für Elemente der Titelseite
\usepackage{tikz} % für Elemente der Titelseite
\usepackage{tabularx} % für Elemente der Titelseite
\usepackage[draft=false,babel,tracking=true,kerning=true,spacing=true,verbose=silent]{microtype} % optischer Randausgleich
\microtypesetup{nopatch={footnote}} % disable footnote microtype warning
% PACKAGES FOR USING THIS PREBUILD STUCTURE
\usepackage{csquotes}   % better quote style for biblatex
\usepackage{graphicx}
%%%%%%%%%%%%%%%%%%%%%%%%%%%%%%%%%_YOUR_PACKAGES_%%%%%%%%%%%%%%%%%%%%%%%%%%%%%%%%%
% UTILITY PACKAGES
\usepackage{cite}
\usepackage{comment} % enables block comments via \begin{comment} ... \end{comment} environment
\usepackage{amsthm} % for definitions, lemmas, etc. - also for defining your own stuff, eg below:
    %\theoremstyle{definition}  % defines a new theorem called definition
    %\newtheorem{definition}{Definition}[section]   % definition setup and call
% IMAGE PACKAGES
\usepackage{wrapfig}    % create figures with wrapped text around it
\usepackage{caption}    % better captions for figures
\usepackage{subcaption} % captions for subfigures
% PRESENTATION PACKAGES
\usepackage{booktabs} % for professional tables
\usepackage{longtable} % for tables over multiple pages
\usepackage{pdflscape} % enables landscape mode for multiple pages in PDF (with longtable)
\usepackage{afterpage} % clears current page by flushing all floats
%%%%%%%%%%%%%%%%%%%%%%%%%%%%%%%%%%%_TITLE_PAGE_%%%%%%%%%%%%%%%%%%%%%%%%%%%%%%%%%%%
\begin{document}

% Beispielhafte Nutzung der Vorlage für die Titelseite (bitte anpassen):
% LaTeX-Vorlage für die Titelseite und Selbständigkeitserklärung einer Abschlussarbeit
% basierend auf der vorigen Institutsvorlage des Instituts für Informatik
% sowie der Vorlage für Promotionsarbeiten.
%%%%%%%%%%%%%%%%%%%%%%%%%%%%%%%%%%%%%%%%%%%%%%%%%%%%%%%%%%%%%%%%%%%%%%%%%%%%%%%%%%%%
% CHANGELOG
% 2014-06-12 Dennis Schneider <dschneid@informatik.hu-berlin.de> (Erweiterung)
% 2023-02-28 Anna Göing <goeingan@informatik.hu-berlin.de> (Strukturhilfe + Packageupdate) 
% 2023-07-14 Florian Willich <florian.willich@informatik.hu-berlin.de> (Formal Revision, Abstract, Package Revision)
%%%%%%%%%%%%%%%%%%%%%%%%%%%%%%%%%%%%%%%%%%%%%%%%%%%%%%%%%%%%%%%%%%%%%%%%%%%%%%%%%%%%
% gepunktete Linie unter Objekt:
\newcommand{\TitelPunkte}[1]{%
  \tikz[baseline=(todotted.base)]{
    \node[inner sep=1pt,outer sep=0pt] (todotted) {#1};
    \draw[dotted] (todotted.south west) -- (todotted.south east);
  }%
}%
% gepunktete Linie mit gegebener Länge:
\newcommand{\TitelPunktLinie}[1]{\TitelPunkte{\makebox[#1][l]{}}}
\makeatletter
%
\newcommand*{\@titelTitel}{Titel der Arbeit}
\newcommand{\titel}[1]{\renewcommand*{\@titelTitel}{#1}} % Titel der Arbeit
\newcommand*{\@titelArbeit}{Arbeitstyp}
\newcommand{\typ}[1]{\renewcommand*{\@titelArbeit}{#1}} % Typ der Arbeit
\newcommand*{\@titelGrad}{akademischer Grad}
\newcommand{\grad}[1]{\renewcommand*{\@titelGrad}{#1}} % Akademischer Grad
\newcommand*{\@titelAutor}{Autor}
\newcommand{\autor}[1]{\renewcommand*{\@titelAutor}{#1}} % Autor der Arbeit
\newcommand*{\@titelGeburtsdatum}{\TitelPunktLinie{2cm}}
\newcommand{\gebdatum}[1]{\renewcommand*{\@titelGeburtsdatum}{#1}} % Geburtsdatum des Autors
\newcommand*{\@titelGeburtsort}{\TitelPunktLinie{5cm}}
\newcommand{\gebort}[1]{\renewcommand*{\@titelGeburtsort}{#1}} % Geburtsort des Autors
\newcommand*{\@titelGutachterA}{\TitelPunktLinie{5cm}}
\newcommand*{\@titelGutachterB}{\TitelPunktLinie{5cm}}
\newcommand{\gutachter}[2]{\renewcommand*{\@titelGutachterA}{#1}\renewcommand*{\@titelGutachterB}{#2}} % Erst- und Zweitgutachter
\newcommand*{\@titelEinreichungsdatum}{\TitelPunktLinie{3cm}} % Datum der Einreichung, wird nicht vom Studenten ausgefüllt
\newcommand*{\@titelVerteidigungsdatum}{} % Verteidigungstext, wird nicht vom Studenten ausgefüllt
\newcommand{\mitverteidigung}{\renewcommand*{\@titelVerteidigungsdatum}{verteidigt am: \,\,\TitelPunktLinie{3cm}}} % Verteidigungsplatzhalter erzeugen
\newcommand*{\@wastwoside}{}
%%
% Titelseite erzeugen:
\newcommand{\makeTitel}{%
	% Speichere, ob doppelseitiges Layout gewählt wurde:
\if@twoside%
	\renewcommand*{\@wastwoside}{twoside}
\else
	\renewcommand*{\@wastwoside}{twoside=false}
\fi
	\KOMAoptions{twoside = false}% Erzwinge einseitiges Layout (erzeugt eine Warnung)
%
	\begin{titlepage}
		% Ändern der Einrückungen
		\newlength{\parindentbak} \setlength{\parindentbak}{\parindent}
		\newlength{\parskipbak} \setlength{\parskipbak}{\parskip}
		\setlength{\parindent}{0pt}
		\setlength{\parskip}{\baselineskip}
		\thispagestyle{empty}
		%
		\begin{minipage}[c][3cm][c]{12cm}
			\textsc{%
				% optischer Randausgleich per Hand:
				\hspace{-0.4mm}\textls*[68]{\Large Humboldt-Universität zu Berlin}\\
				\normalsize \textls*[45]{
					Mathematisch-Naturwissenschaftliche Fakultät\\
					Institut für Informatik
				}
			}
		\end{minipage}
\hfill
		\vfill
		%
		\vspace*{\fill}
		\begin{center}
		\begin{doublespace}
			\vspace{\baselineskip}
			{\LARGE \textbf{\@titelTitel}}\\
			%\vspace{1\baselineskip}
			{\Large
			\@titelArbeit\\
				zur Erlangung des akademischen Grades\\
				\@titelGrad
				\vspace{\baselineskip}
				}
		\end{doublespace}
		\end{center}

		\vfill
\newcolumntype{L}{>{\raggedright\arraybackslash}X}
		{\large \raggedleft
			\begin{tabularx}{\textwidth}{l@{\,\,\raggedright~}L} % verbreiterter Abstand zwischen Feldern wurde gewünscht
				eingereicht von: & \@titelAutor\\
				geboren am: & {\@titelGeburtsdatum}\\
				geboren in: & \@titelGeburtsort
				\vspace{0.5\baselineskip}\\
				Gutachter/innen: & \@titelGutachterA \\
					& \@titelGutachterB
				\vspace{0.5\baselineskip}\\
				eingereicht am: & \@titelEinreichungsdatum \hfill \@titelVerteidigungsdatum
			\end{tabularx}}
			\vspace{-1\baselineskip}\\\phantom{x} % Übler Hack, um eine Warnung wg. einer zu leeren hbox zu verhindern
		% Wiederherstellen der Einrückung
		\setlength{\parindent}{\parindentbak}
		\setlength{\parskip}{\parskipbak}
	\end{titlepage}
%
	% Aufräumen:
	\let\@titelTitel\undefined
	\let\titel\undefined
	\let\@titelArbeit\undefined
	\let\typ\undefined
	\let\@titelGrad\undefined
	\let\grad\undefined
	\let\@titelAutor\undefined
	\let\autor\undefined
	\let\@titelGeburtsdatum\undefined
	\let\gebdatum\undefined
	\let\@titelGeburtsort\undefined
	\let\gebort\undefined
	\let\@titelGutachterA\undefined
	\let\@titelGutachterB\undefined
	\let\gutachter\undefined
	\let\@titelEinreichungsdatum\undefined
	\let\einreichungsdatum\undefined
	\let\@titelVerteidigungsdatum\undefined
	\let\verteidigungsdatum\undefined
%
	\KOMAoptions{\@wastwoside}% Stelle alten Modus (ein-/doppelseitig) wieder her
	\let\@wastwoside\undefined
	\cleardoublepage % ganzes Blatt für die Titelseite
}
% Als Allerallerletztes kommt Selbständigkeitserklärung:
\newcommand{\selbstaendigkeitserklaerung}[1]{%
	%\cleardoublepage% Wieder auf eine eigene Doppelseite
	{\parindent0cm
		\subsection*{Selbständigkeitserklärung}
		Ich erkläre hiermit, dass ich die vorliegende Arbeit selbständig verfasst
		und noch nicht für andere Prüfungen eingereicht habe.
		Sämtliche Quellen einschließlich Internetquellen, die unverändert oder
		abgewandelt wiedergegeben werden, insbesondere Quellen für Texte, Grafiken,
		Tabellen und Bilder, sind als solche kenntlich gemacht. Mir ist bekannt,
		dass bei Verstößen gegen diese Grundsätze ein Verfahren wegen
		Täuschungsversuchs bzw. Täuschung eingeleitet wird.
		\vspace{3\baselineskip}

		{\raggedright Berlin, den #1 \hfill \TitelPunktLinie{8cm}\\}
	}
}%
\makeatother


\titel{Likertshift - an input device for monitoring travel experience while cycling} % Titel der Arbeit
\typ{Bachelorarbeit} % Typ der Arbeit:  Diplomarbeit, Masterarbeit, Bachelorarbeit
\grad{Bachelor of Science (B. Sc.)} % erreichter Akademischer Grad
    % z.B.: Master of Science (M. Sc.), Master of Education (M. Ed.), Bachelor of Science (B. Sc.), Bachelor of Arts (B. A.)
\autor{Max Schlecht} % Autor der Arbeit, mit Vor- und Nachname
\gebdatum{08.06.2000} % Geburtsdatum des Autors
\gebort{Berlin} % Geburtsort des Autors
\gutachter{Prof. Dr. Thomas Kosch}{Prof. Dr. Andrii Matviienko} % Erst- und Zweitgutachter der Arbeit
\mitverteidigung % entfernen, falls keine Verteidigung erfolgt
\makeTitel

\begin{abstract}
	\textbf{Abstract.} Write your abstract here.
\end{abstract}
\newpage

%%%%%%%%%%%%%%%%%%%%%%%%%%%%%%%%_TABLE_OF_CONTENTS_%%%%%%%%%%%%%%%%%%%%%%%%%%%%%%%%
\pagenumbering{gobble}  % page numbers invisible for TOC and filler pages
\tableofcontents
\cleardoublepage    % deactivate for one-sided printing
%\newpage           % activate for one-sided printing
%%%%%%%%%%%%%%%%%%%%%%%%%%%%%%%%%%%%%_CHAPTERS_%%%%%%%%%%%%%%%%%%%%%%%%%%%%%%%%%%%%%
\pagenumbering{arabic}  % start regular page numbers from here
% insert and call your designated chapters here from chapters/... folder

\section{Introduction}\label{section::introduction}

Lorem ipsum dolor sit amet, consetetur sadipscing elitr, sed diam nonumy eirmod
tempor invidunt ut labore et dolore magna aliquyam erat, sed diam voluptua. At
vero eos et accusam et justo duo dolores et ea rebum. Stet clita kasd gubergren,
no sea takimata sanctus est Lorem ipsum dolor sit amet. Lorem ipsum dolor sit
amet, consetetur sadipscing elitr, sed diam nonumy eirmod tempor invidunt ut
labore et dolore magna aliquyam erat, sed diam voluptua. At vero eos et accusam
et justo duo dolores et ea rebum. Stet clita kasd gubergren, no sea takimata
sanctus est Lorem ipsum dolor sit amet. 

I'd like to cite \cite{garraway_genomics-driven_2013}


\section{Motivation}\label{section::motivation}

Lorem ipsum dolor sit amet, consetetur sadipscing elitr, sed diam nonumy eirmod
tempor invidunt ut labore et dolore magna aliquyam erat, sed diam voluptua. At
vero eos et accusam et justo duo dolores et ea rebum. Stet clita kasd gubergren,
no sea takimata sanctus est Lorem ipsum dolor sit amet. Lorem ipsum dolor sit
amet, consetetur sadipscing elitr, sed diam nonumy eirmod tempor invidunt ut
labore et dolore magna aliquyam erat, sed diam voluptua. At vero eos et accusam
et justo duo dolores et ea rebum. Stet clita kasd gubergren, no sea takimata
sanctus est Lorem ipsum dolor sit amet. Lorem ipsum dolor sit amet, consetetur
sadipscing elitr, sed diam nonumy eirmod tempor invidunt ut labore et dolore
magna aliquyam erat, sed diam voluptua. At vero eos et accusam et justo duo
dolores et ea rebum. Stet clita kasd gubergren, no sea takimata sanctus est
Lorem ipsum dolor sit amet.   

Duis autem vel eum iriure dolor in hendrerit in vulputate velit esse molestie
consequat, vel illum dolore eu feugiat nulla facilisis at vero eros et accumsan
et iusto odio dignissim qui blandit praesent luptatum zzril delenit augue duis
dolore te feugait nulla facilisi. Lorem ipsum dolor sit amet,


%%%%%%%%%%%%%%%%%%%%%%%%%%%%%%%%%%%_BIBLIOGRAPHY_%%%%%%%%%%%%%%%%%%%%%%%%%%%%%%%%%%%
% create your bibliography based on your files in library/...
% remember to edit \addbibresource in the TEMPLATE_PACKAGSES area above!
\newpage
\pagenumbering{roman} % start roman page numbers from here (optional)
\bibliographystyle{abbrv}
\bibliography{references.bib}
%%%%%%%%%%%%%%%%%%%%%%%%%%%%%%%%%%%%%_APPENDIX_%%%%%%%%%%%%%%%%%%%%%%%%%%%%%%%%%%%%
\section*{Appendix} \label{Appendix}
\addcontentsline{toc}{section}{Appendix}    % adds entry to table of contents
\selbstaendigkeitserklaerung{\today}
%\input{chapters/xxx}                       % add in case you have additional images/tables
\end{document}
%%%%%%%%%%%%%%%%%%%%%%%%%%%%%%%%%%%%%%%%%%%%%%%%%%%%%%%%%%%%%%%%%%%%%%%%%%%%%%%%%%%%
