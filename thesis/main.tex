%%%%%%%%%%%%%%%%%%%%%%%%%%%%%%%%%_TEMPLATE_PACKAGES_%%%%%%%%%%%%%%%%%%%%%%%%%%%%%%%%%
\documentclass[
	a4paper,
	pagesize,
	pdftex,
	12pt,
	% twoside,    % + BCOR darunter: für doppelseitigen Druck aktivieren, sonst beide deaktivieren
	% BCOR=5mm,   % Dicke der Bindung berücksichtigen (Copyshop fragen, wie viel das ist)
	english,
	fleqn,
	final,
	]{scrartcl}
    
% PACKAGES FOR THE INSTITUTSVORLAGE
\usepackage[utf8]{inputenc}
\usepackage[english]{babel}
\usepackage[unicode=true,hidelinks]{hyperref} % label, references, websites and crosslinks in PDF
\usepackage{setspace} % für Elemente der Titelseite
\usepackage{tikz} % für Elemente der Titelseite
\usepackage{tabularx} % für Elemente der Titelseite
\usepackage[draft=false,babel,tracking=true,kerning=true,spacing=true,verbose=silent]{microtype} % optischer Randausgleich
\microtypesetup{nopatch={footnote}} % disable footnote microtype warning
% PACKAGES FOR USING THIS PREBUILD STUCTURE
\usepackage{csquotes}   % better quote style for biblatex
\usepackage{graphicx}
%%%%%%%%%%%%%%%%%%%%%%%%%%%%%%%%%_YOUR_PACKAGES_%%%%%%%%%%%%%%%%%%%%%%%%%%%%%%%%%
% UTILITY PACKAGES
\usepackage{cite}
\usepackage{comment} % enables block comments via \begin{comment} ... \end{comment} environment
\usepackage{amsthm} % for definitions, lemmas, etc. - also for defining your own stuff, eg below:
    %\theoremstyle{definition}  % defines a new theorem called definition
    %\newtheorem{definition}{Definition}[section]   % definition setup and call
% IMAGE PACKAGES
\usepackage{wrapfig}    % create figures with wrapped text around it
\usepackage{caption}    % better captions for figures
\usepackage{subcaption} % captions for subfigures
% PRESENTATION PACKAGES
\usepackage{booktabs} % for professional tables
\usepackage{longtable} % for tables over multiple pages
\usepackage{pdflscape} % enables landscape mode for multiple pages in PDF (with longtable)
\usepackage{afterpage} % clears current page by flushing all floats
%%%%%%%%%%%%%%%%%%%%%%%%%%%%%%%%%%%_TITLE_PAGE_%%%%%%%%%%%%%%%%%%%%%%%%%%%%%%%%%%%
\begin{document}

% Beispielhafte Nutzung der Vorlage für die Titelseite (bitte anpassen):
\input{Institutsvorlage}

\titel{Likertshift - an input device for monitoring Travel Experience while cycling} % Titel der Arbeit
\typ{Bachelorarbeit} % Typ der Arbeit:  Diplomarbeit, Masterarbeit, Bachelorarbeit
\grad{Bachelor of Science (B. Sc.)} % erreichter Akademischer Grad
    % z.B.: Master of Science (M. Sc.), Master of Education (M. Ed.), Bachelor of Science (B. Sc.), Bachelor of Arts (B. A.)
\autor{Max Schlecht} % Autor der Arbeit, mit Vor- und Nachname
\gebdatum{08.06.2000} % Geburtsdatum des Autors
\gebort{Berlin} % Geburtsort des Autors
\gutachter{Prof. Dr. Thomas Kosch}{Prof. Dr. Andrii Matviienko} % Erst- und Zweitgutachter der Arbeit
\mitverteidigung % entfernen, falls keine Verteidigung erfolgt
\makeTitel

\begin{abstract}
    \subsubsection*{Abstract}
    This work explores creating an improved method for recording ground truth Travel Experience data while cycling.
    It shows the design process of a physical prototype, resembling a standard bicycle twist gearshift, which allows cyclists to provide ratings on a standard five-point Likert scale without interrupting their cycling or compromising their safety.
    A field study was conducted to evaluate using the prototype against other state-of-the-art methods.
    Results showed that participants strongly preferred using the prototype, highlighting its ease of use, intuitive handling, and lack of concerns regarding social acceptance or additional time requirements compared to the other methods.
    Additionally, an evaluation of the recorded ground truth Travel Experience data confirmed that the data collected using the prototype was of comparable or superior quality to that collected using the existing state-of-the-art methods.
    This research paves the way for larger field studies by providing a better method to either directly record cyclists' Travel Experience or the necessary ground truth data to infer it from additional sensory data.
\end{abstract}

\newpage

%%%%%%%%%%%%%%%%%%%%%%%%%%%%%%%%_TABLE_OF_CONTENTS_%%%%%%%%%%%%%%%%%%%%%%%%%%%%%%%%
\pagenumbering{gobble}  % page numbers invisible for TOC and filler pages
\tableofcontents
\cleardoublepage    % deactivate for one-sided printing
%\newpage           % activate for one-sided printing
%%%%%%%%%%%%%%%%%%%%%%%%%%%%%%%%%%%%%_CHAPTERS_%%%%%%%%%%%%%%%%%%%%%%%%%%%%%%%%%%%%%
\pagenumbering{arabic}  % start regular page numbers from here
% insert and call your designated chapters here from chapters/... folder

\newpage\section{Motivation}\label{sec:motivation}

\begin{itemize}
    \item Motivation for recording Travel Experience / Satisfaction
    \item Use cases \cite{what_if_your_car_would_care}
    \item Different methods
    \item Emotion recognition \cite{facial_emotion_recognition} \cite{vemotion} \cite{affective_automotive_user_interfaces} \cite{happy_or_scared} \cite{emotion_sensing_for_ebike_safety} \cite{driver_emotion_recognition_survey} \cite{towards_empathetic_car_interfaces}
    \item Differentiation of "ground-truth-methods"
\end{itemize}

\newpage\section{Related Work}\label{sec:related_work}

\subsection{Methods for Recording Cyclists' Subjective Experiences}

\cite{cycling_subjective_experience}

\subsection{Digital Control Devices for Bicycles}

\cite{brotate_and_tribike}

\subsection{Research Questions}

\newpage\section{Prototype Design}\label{sec:prototype_design}

\subsection{Design Requirements}

\begin{itemize}
    \item cheap/easy to reproduce
    \item using standard components
    \item using widely available "construction methods" / machines
    \item end user safety \cite{gesturing_on_the_handlebars} \cite{no_need_to_stop} \cite{text_me_if_you_can} \cite{brotate_and_tribike}
\end{itemize}

\subsection{Mechanical Design}

\begin{itemize}
    \item potentiometers vs. rotary switches vs. buttons
    \item brotate and tribike \cite{brotate_and_tribike}
    \item advantages/disadvantages of using our own hardware design (cost, reliability, availability, obsolete components)
\end{itemize}

\subsection{Electrical Design and Firmware}

\begin{itemize}
    \item BLE
    \item MCU choice (softdevices conformity)
    \item battery runtime analysis
\end{itemize}

\subsection{Frontend Design}

\begin{itemize}
    \item reusability
    \item BLE interfacing (communication "protocol")
    \item language/framework choice?
    \item UI design?
\end{itemize}

\newpage\section{Study Design}\label{sec:study_design}

We decided to evaluate our prototype in a semi-naturalistic field-study, comparing it against similar methods, capable of recording \CSE for individual segments of a larger route.
This section describes how we went about designing this study.

\subsection{Recording Methods}\label{subsec:recording_methods}

Our method using the “LikertShift” device works by recording its currently selected value, whenever we receive a new GPS location.
Thus, a selected value remains valid until the users selects a new one, and we receive their next GPS location, so a new rating gets recorded whenever they notice a change in their subjective experience and remains valid, onwards from that point in time, until they adjust the rating again.
This allows users to choose a rating frequency they consider appropriate by themselves.

We selected two methods from our survey of the related work we regard as suitable to compare against our method and adjusted them, to allow for a fair comparison.
The first one, referred to as “Audio Recording” from now on, is taken from the work of \citetext{evaluation_models_for_cyclists_perception}{Yamanaka et al.}, who captured audio data while cycling and required participants to rate pre-defined route-segments using multiple measures on scales from 1 to 5.
We adjusted it to only used one measure and also let participants perform ratings on arbitrary route segments, in contrast to pre-defined ones.
The second method, referred to as “Mapping” is the one proposed by \citetext{using_mental_mapping}{Manton et al.}, who let participants color code segments of a previously driven route, based on their risk assessment.
In our study, we provided participants with a printed map, picturing the route they drove and let them divide it into multiple segments themselves, assigning a numerical rating to each segment, instead of color coding it.

\subsubsection*{Choosing a Measure}

We had to take great care to select a suitable measure, so we could compare the data collected by different participants and the different methods against each other.
This is inherently difficult, as cyclists' subjective experiences are arguably very subjective and will thus vary drastically from participant to participant.

As mentioned in the introduction to \autoref{sec:related_work}, most works rely on risk/safety and stress/comfort measures to quantify \CSE.
The advantage of using risk/safety would be its independence from many external factors and high dependence on the chosen route.
Constructing a route with highly varying risk factors would be comparatively easy, but deliberately placing participants in high-risk environments would be extremely unethical, which led us to quickly dismiss that approach.
Letting participants rate their comfort level or travel satisfaction comes with its own set of challenges though.
Overall comfort level depends on a lot of actors, such as participants initial mood, their experience in riding a bicycle, but also external factors such as weather, traffic, or surrounding scenery of the route.
To be able to properly compare our recorded datasets against each other, we needed to find a way to eliminate as many of these factors as possible.

As findings by \citetext{thinking_aloud_on_the_road}{McIlroy et al.} revealed the high impact of road surface quality on cyclists overall comfort and satisfaction with their travel, we decided to limit our measure to this, using the metric of “Travel satisfaction, based on the road” as the \CSE measure that participants have to use to rate segments of the route.
We carefully explained this metric to each participant before conducting the study, instructing them on what should (road condition, road type, available space, slope of the road, etc.) and should not (general mood, current traffic situation, route scenery, etc.) affect their ratings.
Ratings were to be performed on a Likert scale from 1 to 5, denoting high dissatisfaction and high satisfaction, respectively.
This measure should exert a similar mental demand on participants, as they still need to feel out their comfort level, while increasing data quality, allowing us to perform quantitative comparisons of the different methods used.

\subsection{Study Routes}

\begin{itemize}
    \item Length
    \item Features
\end{itemize}

\subsection{Data Collection}

\subsubsection{Route Data}

\subsubsection{Questionnaires}

\subsubsection{Interview}

\subsection{Procedure}

\subsubsection{Safety Precautions}

\newpage\section{Results}\label{sec:results}

\subsection{Participants}

\begin{itemize}
    \item demographics
    \item snowball sampling
    \item compensation
    \item cancellations
    \item length (time)
\end{itemize}

\subsection{Recorded Data Evaluation}

\subsubsection{Process / Method}

\subsection{Questionnaires and Interviews}

\subsubsection{Coding}

\newpage\section{Discussion}\label{sec:discussion}

\subsection{Research Questions}

\subsection{Design Requirements}\label{subsec:discussion_design_requirements}

\subsection{Future Work}

\begin{itemize}
    \item hybrid methods (manual mapping)
    \item missing software qol features
    \item longer studies
\end{itemize}

\newpage\section{Conclusion}\label{sec:conclusion}

In this work, we described the successful design and construction of the \likertshift, a prototype device that can be used to record cyclists' subjective experiences, as well as its evaluation against other state-of-the-art methods in a semi-naturalistic field-study.
Although quantitative analysis of recorded data revealed little to no statistical evidence supporting a measurable improvement in data quality using our developed \likertshift method, qualitative assessments of participants opinions regarding the different methods provided valuable insights.
They highlighted the perceived attractiveness, intuitiveness, and efficiency of using our physical device.
They also voiced concerns about the social acceptance of speech recording during cycling and the high effort and increased mental demand associated with retrospective mental mapping approaches that rely on the memorization of route segments, particularly for longer routes.
From our collected evidence, we see a great potential in using physical devices for recording data on cyclists' subjective experiences, especially in longer, more naturalistic field-studies.


%%%%%%%%%%%%%%%%%%%%%%%%%%%%%%%%%%%_BIBLIOGRAPHY_%%%%%%%%%%%%%%%%%%%%%%%%%%%%%%%%%%%
% create your bibliography based on your files in library/...
% remember to edit \addbibresource in the TEMPLATE_PACKAGSES area above!
\newpage
\pagenumbering{roman} % start roman page numbers from here (optional)
\bibliographystyle{abbrvurl}
\bibliography{references.bib}
%%%%%%%%%%%%%%%%%%%%%%%%%%%%%%%%%%%%%_APPENDIX_%%%%%%%%%%%%%%%%%%%%%%%%%%%%%%%%%%%%
\section*{Appendix} \label{Appendix}
\addcontentsline{toc}{section}{Appendix}    % adds entry to table of contents
\selbstaendigkeitserklaerung{\today}
%\input{chapters/xxx}                       % add in case you have additional images/tables
\end{document}
%%%%%%%%%%%%%%%%%%%%%%%%%%%%%%%%%%%%%%%%%%%%%%%%%%%%%%%%%%%%%%%%%%%%%%%%%%%%%%%%%%%%
